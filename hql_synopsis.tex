\documentclass[11pt]{article}
%---- defitions ----
\def\Author{Andreas Bock, bock@andreasbock.dk\\
Johan Astborg, joastbg@gmail.com\\\\
Supervisors:\\
Jost Berthold, jb.diku@gmail.com\\
Sinan Gabel, sinan.gabel@gmail.com
}
\def\Title{\bf Project Synopsis\\ HQL - \textsc{Hiperfit} Quant Library}
% <Add more defitions here>
%-------------------

%---- packages ----
\usepackage[]{amsmath}
\usepackage[english]{babel}
\usepackage[utf8]{inputenc}
\usepackage{graphicx}
\usepackage{moreverb}
\usepackage{hyperref}

%---- settings ----
\topmargin=-1.1in % start text higher on page
\textheight=695pt
\usepackage[T1]{fontenc} % font
%\renewcommand*{\familydefault}{\sfdefault}
\setlength{\parindent}{0in}
%-------------------

\begin{document}
\title{\Title}
\author{\Author}
\date{\today}
\maketitle

\begin{abstract}

% Something about how pricing is done now, and what we will do

\end{abstract}

\section*{Problem Statement}

Our \href{http://hiperfit.dk/studentproject_haskell_library_finance.html}{project}, proposed by the HIPERFIT research center, will construct the core
of a modular Haskell library for quantitative finance.
The main focus is to work on code structure and code architecture to reach a proof-of-concept level.

\section*{Elaboration}

% Description of the domain, used to be in the abstract
In finance, a portfolio is a collection of assets such as equity, bonds or other
types of cash flows. The present value of such a collection can be computed through
discounting to reflect the value they would have if they existed today, allowing
investors to evaluate if their resources could yield higher returns elsewhere.
Financial instruments are priced differently, some using closed-form functions and
others relying on stochastic simulation.\\

% Motivation for our project
    % Recent changes in the financial industry
    % Why a functional language is a suitable choice

The recent financial crisis has emphasized the need ...\\ % Currently reading articles to substantiate these claims (check HIPERFIT publications)

A functional language such as Haskell has a number of attractive properties that
makes it suitable for the financial domain.\\ % Develop this

% Elaboration of the project description
    % Exploit Haskell type-system to produce fail-safe code (at least at runtime)
    % Extendability -> focus on making clever classes and types 
    % Parallelism -> doesn't seem to be _that_ important, but it's nice to keep in mind
    % Extendability -> focus on making clever classes and types 

We plan on designing the architecture for a Haskell library for quantitative finance.
Our focus will be on exploring how Haskell's advanced type system can be exploited
to ensure safe and correct code.
We will also explore the possibility of incorporating parallelism to cater for the
demands that the industry has on performance.\\

% Mathematica "port"
Finally, library will be heavily inspired by the {\tt DerivativesExpert} package developed by Sinan Gabel for the \emph{Mathematica} software\cite{Mathematica:DerivativesExpert}.

\section*{Learning Goals}

The project will teach the students the following:

\begin{enumerate}
\item The student will be able to design and implement medium to large scale software systems in a functional language. % arch
\item The student will be able to identify and use features of a functional language to enforce the necessary correctness needed in financial computing. % correctness
\item The students will be able to describe the concepts of financial valuation and its relevance to portfolio management. % finance
\item The students will be able to develop a sustainable and extendable prototype Haskell library for pricing common financial products. % concrete result
\end{enumerate}

\section*{Limitations}

% 1) Not a fully-fledged library like DE or Quantlib
Firstly, we will not expect to produce a fully-fledged library like {\tt DerivativesExpert} or Quantlib\cite{Ame2003}.

% 2) Benchmarking or comparisons (?)
Secondly, and partly due to the point above, we will not perform an in-depth survey
comparing the result of our project with similar ones such as the aforemetioned libraries.

% 3)

% 4)

\section*{Possible additions}

\section*{Schedule}

% REFERENCES
\bibliographystyle{abbrv}
%\addcontentsline{toc}{section}{References}
\bibliography{hql}

\end{document}
