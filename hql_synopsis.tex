\documentclass[11pt]{article}
%---- defitions ----
\def\Author{Andreas Bock, bock@andreasbock.dk\\
Johan Astborg, joastbg@gmail.com\\\\
Supervisors:\\
Jost Berthold, jb.diku@gmail.com\\
Sinan Gabel, sinan.gabel@gmail.com
}
\def\Title{\bf Project Synopsis\\ HQL - \textsc{Hiperfit} Quant Library}
% <Add more defitions here>
%-------------------

%---- packages ----
\usepackage[]{amsmath}
\usepackage[english]{babel}
\usepackage[utf8]{inputenc}
\usepackage{graphicx}
\usepackage{moreverb}
\usepackage{hyperref}

%---- settings ----
\topmargin=-1.1in % start text higher on page
\textheight=695pt
\usepackage[T1]{fontenc} % font
%\renewcommand*{\familydefault}{\sfdefault}
\setlength{\parindent}{0in}
%-------------------

\begin{document}
\title{\Title}
\author{\Author}
\date{\today}
\maketitle

\begin{abstract}

% Something about how pricing is done now, and what we will do

\end{abstract}

\section*{Introduction}

% Something introductory (you don't say!) 
The financial sector has become a focal point of interest...

% Open source -> do banks want a uniform quant platform..? cite if possible
Things have changed and new requirements and technologies are pushing the needs
of a redesign in a more open language. Also, insights into source code and the
details behind an implementation is something the market asks for.

% Algorithms
Automated algorithmic trading systems have low operational risk due to the lack
of human contact, but require software correctness to work. An otherwise feasible
trading model to wreak havoc on markets because of a software error.

% Safeguards are also compromised
Further, brokerages and exchanges attempt to lower operational risk by implementing
safeguards to make it difficult for human error to affect markets (e.g. fat-fingering
trades\footnote{Human error causing larges trades to be made that create sudden imbalances in the markets}). Albeit thoroughly tested, bugs are ubiquitous and
these safeguards may be just as prone to them. 

% Legislation, cite new Danish regulations?
The financial crisis of 2008 also caused legislators to take a more conservative
stance on risk, as the collapse on Lehman Brothers reverberated throughout
the world's economies.

Consequently, financial institutions' risk management tools must become more
sophisticated, putting stress on the quality of the software.

By designing and building correct and safe software we may hope to eliminate some
of the unnecessary risk in financial modelling.

\section*{Problem Statement}

In this project we will design the software architecture for a Haskell library for quantitative finance.
The desired functionality is defined by an existing Mathematica library {\tt DerivativesExpert}, which
we will re-engineer into Haskell.
The main objective is to use the Haskell programming language for design and architecture and reach a proof-of-concept level.
The project is to be completed within the HIPERFIT research center.

\section*{Elaboration}

In finance, a portfolio is a collection of assets such as equities, bonds and other
types of cash flows. The present value of such a collection can be computed through
discounting to reflect the theoretical value they would have if they existed today. This allows
investors to evaluate if their resources could yield higher returns elsewhere.
Financial instruments are priced differently, some using closed-form functions and
others relying on stochastic simulation. In this project, we'll mainly work with bonds and
time value of money.\\

Functional languages lend themselves well to the domain of finance due
to several reasons. Firstly, functions are first-class citizens, and programs are constructed
from functions entirely. Because of the mathematical nature of finance, a functional
language preferable and makes it easier to write and test mathematical expressions.\\

Haskell is a pure language meaning that a function will always return the same value
if given the same input\footnote{with exceptions such as {\tt IO}}. This makes it ideal for
financial computations where we want to be certain our computations are not changed by some
global state.\\

%A functional language such as Haskell has many attractive properties that
%makes it a perfect fit for finance. This means that functions are first class citizens,
%and programs are constructed from functions entirely. Because of the mathematical nature of finance, a functional
%language preferable and makes it easier to write and test mathematical expressions.

Additionally, Haskell is a strongly typed language with type inference. In other words, 
types and type classes make up a large portion of the language design. A strongly typed language has the
advantage of finding many bugs at compile time (rephrase?), and the class system enables us to overload
functionality in our programs based on our types. This is a very desirable feature, as we want our financial products
to share the same interface (e.g. computing present value, expiration date) but with slightly different implementations.\\

Finally, functional languages can easily preserve the inherent parallelism of computations 
making them ideal for performance-critical tasks such as pricing financial products or
computing risk.\\

In summary, our focus will be on exploring how Haskell's advanced language features can be used
to ensure safe and correct code, without compromising a good extendable software architecture.

% Mathematica "port"
%Finally, the library will be heavily influenced by the {\tt DerivativesExpert} package developed by Sinan Gabel for the \emph{Mathematica} software\cite{Mathematica:DerivativesExpert}.

\section*{Learning Goals}

Learning goals and objectives:

\begin{enumerate}
\item The student will be able to design and implement medium to large scale software systems in a functional language. % arch
\item The student will be able to identify and use features of a functional language to enforce the necessary correctness needed in financial computing. % correctness
\item The students will be able to describe the concepts of financial valuation and its relevance to portfolio management. % finance
\item The students will be able to develop a sustainable and extendable prototype Haskell library for pricing common financial products. % concrete result
\item To translate the functionality of existing software and re-engineer it in a functional language.
\end{enumerate}

\section*{Limitations}

% 1) Not a fully-fledged library like DE or Quantlib
Firstly, we will not expect to produce a fully-fledged library like {\tt DerivativesExpert} or Quantlib\cite{Ame2003}.

% 2) Benchmarking or comparisons (?)
Secondly, and partly due to the point above, we will not perform an in-depth survey
omparing the result of our project with similar ones such as the aforemetioned libraries.

% 3)

This project is mainly limited by time, which is a constraint derived from the course format in which the projects is held.
Further, the project is somewhat limited to the existing code and requirements encapsulated in DerivativesExpert.
The scope has been narrowed to fit the size of the project, and will mainly include pricing functionality for bonds.

\section*{Possible additions}

\section*{Schedule}

% REFERENCES
\bibliographystyle{abbrv}
%\addcontentsline{toc}{section}{References}
\bibliography{hql}

\end{document}
